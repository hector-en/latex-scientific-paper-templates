\def\introduction {
\lettrine[lines=2]{\color{color2}M}ycobacteria{}\textit{} 
 have a characteristic highly impermeable cell wall composed of around 80 mycolic acids forming a first carbon layer, which is covalently linked to the inner peptidoglycan layer. The linker consists of an arabinogalactan polysaccharide middle layer of three arabinan chains attached to the linear galactan polymer at position 5 (\fullref{fig:figure1}). Mycolic acids are synthesized in an anabolic bacterial fatty acid pathway (\fullref{fig:figure2}), that elongates fatty acyl chains in an enzyme-catalyzed post-translational process. It involves the production of short-chain primers by the FAS I (Rv2524c) protein which are attached to the mycolic acid-specific carrier protein AcpM by the ß-ketoacyl ACP synthase III and FanH enzymes. These primers are then transported to the repetitive FAS II enzyme system which then extends the carbons. Extension is completed by the 2-trans enoyl ACP reductase InhA (Rv1484) \cite{Clifton2007}. This rigid structure and the high amount of long fatty acid chains in the cell wall cause a low cell membrane permeability. High resistance to antibiotic molecules is therefore a consequence of a rigid cell wall, an active efflux of antibiotic molecules and chemical modification i.e., hydrolysis in ß-lactams \cite{Jarlier1994}.\\[1em]

{\footnotesize\mycoliclayer{0.9}{\small Cell wall structure of \textit{M. smegmatis} mc$^2$155. Long chain mycolic acids (red), 3 chains of arabinofuran (blue), a galactofuran layer (purple)  covalently linked to the peptidoglycan layer (green). Figure was adopted from \cite{Clifton2007} }}\\[1em]
\twocolend
    {\footnotesize \synthesis{0.8}{\small Bacterial fatty acid synthesis type II (FAS II). Isoniazid and Ethionamide target the InhA(Rv1484) and KatA(Rv2245) enzymes (red). Inhibiting the completion of the Fas II cycle. Figure was adopted from \cite{Clifton2007} }}
\twocolstart

\textit{M. smegmatis}  mutant mc$^2$155,  is a non-pathogenic, fast growing model for the study of mycobacteria. It is a saprophytic mycobacterium (does not enter the epithelial cells or remains inside phagocytes) \cite{Reyrat2001}. It therefore does not present the pathogenicity of \textit{M. tuberculosis}. However, it is a good model to study antibacterial agents targeting the cell wall of \textit{M. tuberculosis}, since both have the same architecture and physiology. This property also makes it a good model for genetic modification and cellular studies in mycobacteria in general \cite{Sparks2023}. Ultimately, it is considered evolutionary too distant from the TB causing  \textit{M. tuberculosis} and is generally not a good model for virulence or pathogenicity studies on TB \cite{Reyrat2001}. 
Isoniazid (INH) and the related Ethionamide (ETH) target the same enzymes in the FAS II system of the mycolic acid synthesis: InhA (\fullref{fig:figure2}Rv1484) and ß-ketoacyl-ACP synthase KasA (\fullref{fig:figure2}Rv2245/46). Both are prodrugs and require intra-cellular activation by katG (INH) or ethA (ETH) in the presence of NAD. Activation of ETH can additionally be regulated by ethR.
Both exhibit a narrow spectrum bactericidal effect on fast-growing mycobacteria by disrupting the type II mycolic acid synthesis in the cell wall synthesis \cite{Banerjee1994,York2008}.
\textit{M.tuberculosis} and \textit{M. smegmatis} are acid-fast bacilli (AFB). Their thick, hydrophobic cell wall confers the ability to resist decolorisation by acids during staining. A common technique is the carbol-fuchsin method also referred as the Ziehl-Neelsen staining (ZN-staining), to form a mycolate-fuchsin complex in the cell wall resulting in AFB appearing red under the microscope, while the rest of the stain loses the fuchsin component due to the missing protective mycolic acid cell wall and can be counter stained for an effective visual contrast \cite{Bayot2022}. We investigate the effect of INH, ETH and a novel chemical entity (NCE) on the the growth and morphology of \textit{M.smegmatis} by verifying acid fastness using a cold ZN-staining method followed by bright-field microscopic examination.

}
\endinput