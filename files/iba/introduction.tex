\def\introduction {
\lettrine[lines=2]{\color{color2}M}ycobacteria{}\textit{} 
 have a characteristic highly impermeable cell wall composed of around 80 mycolic acids forming a first carbon layer, which is covalently linked to the inner peptidoglycan layer. The linker consists of an arabinogalactan polysaccharide middle layer of three arabinan chains attached to the linear galactan polymer at position 5 (\fullref{fig:figure1}). Mycolic acids are synthesized in an anabolic bacterial fatty acid pathway (\fullref{fig:figure2}), that elongates fatty acyl chains in an enzyme-catalyzed post-translational process. It involves the production of short-chain primers by the FAS I (Rv2524c) protein which are attached to the mycolic acid-specific carrier protein AcpM by the ß-ketoacyl ACP synthase III and FanH enzymes. These primers are then transported to the repetitive FAS II enzyme system which then extends the carbons. Extension is completed by the 2-trans enoyl ACP reductase InhA (Rv1484) \cite{Clifton2007}. This rigid structure and the high amount of long fatty acid chains in the cell wall cause a low cell membrane permeability. High resistance to antibiotic molecules is therefore a consequence of a rigid cell wall, an active efflux of antibiotic molecules and chemical modification i.e., hydrolysis in ß-lactams \cite{Jarlier1994}.\\[1em]

{\footnotesize\mycoliclayer{0.9}{\small Cell wall structure of \textit{M. smegmatis} mc$^2$155. Long chain mycolic acids (red), 3 chains of arabinofuran (blue), a galactofuran layer (purple)  covalently linked to the peptidoglycan layer (green). Figure was adopted from \cite{Clifton2007} }}\\[1em]
\twocolend
    {\footnotesize \synthesis{0.8}{\small Bacterial fatty acid synthesis type II (FAS II). Isoniazid and Ethionamide target the InhA(Rv1484) and KatA(Rv2245) enzymes (red). Inhibiting the completion of the Fas II cycle. Figure was adopted from \cite{Clifton2007} }}
\twocolstart

}
\endinput