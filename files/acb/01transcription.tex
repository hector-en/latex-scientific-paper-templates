\def\topicTranscription {
\section{RNA Polymerase Transcription}
% --------------------------------------------------------------------------------------------------------------------------------------------
%                                                      Transcription & Regulation (General)
% --------------------------------------------------------------------------------------------------------------------------------------------
\subsection{\underline{Transcription \& Regulation (General)}}
% Set the width of each column
 %
\begin{longtable}{p{1in}p{4.8in}}
\textbf{Transcription}(Fg1) & 
    Polymerases transcribes DNA in 6 steps and in the presents of four building blocks, which need to be present: \textbf{RNA} Polymerase, NTP's, DNA, GTFs (inititators) and \textcolor{blue}{in \textbf{Pol II}Activators and Co-activators}. Moreover, RNA Polymerase transcriptioin can not transcribe the DNA without the GTFs. Therefore activity is regulated by the TF and co-activators~\citep{Wk4video3}.\\
    
\textbf{Subunits}~(Fig2).& 
    RNA Polymerases have between \textbf{12 - 17 subunits} (Pol I: 14; Pol II: 12; Pol III: 17).\\

\textbf{GFTs}~(Fig2). & 
    GTF are responsable for stabilising the initiation and open complex are integral in RNA Plo I and III structure but recruited by co-activators in Pol II (TfI/II$_{\alpha,\beta}$). While GTF responsible  for the DNA binding are shared accross the three Polymerases, DNA opening and start site selection are specific to RNA Polymerases~\citep{Wk4video3}.\\
\end{longtable}\\[2em]
 % --------------------------------------------------- Regulation (General) ------------------------------------------------------------------
\begin{longtable}{p{1in}p{4.8in}}
\textbf{\textcolor{blue}{Regulation (General)}} &
    - Regulation of Polymerase transcription is achived by
    \begin{itemize}
        \item Any catabolic signaling pathway (growth and stress- signaling)
        \item Epigenetic control through DNA compaction in nucleosomes. 
    \end{itemize}. 
    - The human genome has non-coding genes that regulate transcription by splicing mechanisms.
    - Diseases, that can occur because of failed regulation include: Cancer, diabetis,...~\citep{Wk4video3}
    
\end{longtable}
% --------------------------------------------------------------------------------------------------------------------------------------------
%                                                      Transcription & regulation Pol I/III
% --------------------------------------------------------------------------------------------------------------------------------------------

\subsection{\underline{Recruitment \& Regulation (Pol I/III} }
\begin{longtable}{p{1in}p{4.8in}}
\textbf{Pol I}~(Fig3) & 
    After TF signaling, the Upstream Binding Factor \textbf{(UBF)} bind to the core Promoter \textbf{(CP)} and Upstream Control Element\\ 

\textbf{Pol III}~(Fig4) &  
    RNA Polymerase III transcribes untranslated RNA (tRNA, 5S rRNA, U6 snRNA,...) and RNAs shorter than 350nt through two highly distinct features: \textbf{facilitated reinitiation}(binds repeatably to the same gene without disassembling steps) and \textbf{termination with Polu-U} (termination only requires a strech of Poly-T on the template strand). Theses properties allow Pol III to achieve a very high levels of RNA transcription. Recruitement of Pol III is only possible with the nucleosome -1 upstream the promoter region~\citep{Wk4video3}.\\
%
% --------------------------------------------Regulation (Pol I/III) -------------------------------------------------------------------------
\textbf{\textcolor{blue}{Regulation (Pol I/III)}}( Fig4) &
    Pol I and III are involved in ribosaomal RNA biogenesis (houskeeping in Pol III) and are therefore part of anabolic pathways. Any catabolic signaling pathway will therefore also regulate transcription. In Pol I transcription is affected especially by MAPK,PI3K, mTOR abd c-Jun in Pol III mTOR and Maf1(inhibitor) are the major growth factors and c-Myc and CK2 the main tumour suppressors~\citep{Wk4video3}.TBP (TATA binding TF) is a major interaction site in the  PIC for TF that inhibit TBP binding to DNA and TFII-B\citep[p.466]{pol2}
\\
\end{longtable}
% --------------------------------------------------------------------------------------------------------------------------------------------
%                                                      Transcription & regulation Pol II
% --------------------------------------------------------------------------------------------------------------------------------------------
\subsection{\underline{Recruitment \& Regulation (Pol II)} }

\begin{longtable}{p{1in}p{4.8in}}
\textbf{Pre-initiation}(Fig5) & \\
& The preinitiation complex (PIC) is composed of 
\begin{itemize}
    \item an upstream DNA (uDNA) part: TBP (TATA binding protein), TFII-A,B,F Bound to uDNA (Promoter sequence [TSS;250bp])
    \item and a Downstream DNA (dDNA) Region:   TFII-H,D Bound to dDNA (promoter sequence [50bp;TSS])
\end{itemize}\\
& Transcription can only proceed when all Transcription factors TFII-B,E,F  are bound to RNA Pol II  and the full  promoter ssDNA template sequence of TSS[50;250]bp is been pulled into the Pol II cleft. For this, the preinitiation complex (PIC) needs to be assembled, since it is the main site for RNA Pol II regulation via TFs. Transcription starts when TBP receives a signal to initiate the assembly of the PIC by binding to the TATA sequence. Factors, which could inhibit binding of the TBP to the TATA box aoud be TAF11/13 and TAF1 or TAND. They would prevent TFII-A,B binding to TBP and inhibit therefore inhibit PIC assembly \citep[p.466]{pol2}.   \\
\textbf{Pol II} & \\ 
%
% ----------------------------------------------- Regulation (Pol II) -------------------------------------------------------------------------
\textbf{\textcolor{blue}{Regulation}} &
%
\\
\end{longtable}


}
\endinput