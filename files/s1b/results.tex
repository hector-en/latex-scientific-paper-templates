\def\results {
The DNA parts concentrations were 200 \text{ng}/u\text{L} and 
50 \text{ng}/u\text{L} for Kan-Ori. Linker ligation ($20~^\circ\mathrm{C}$) and restiction digestion ($37~^\circ\mathrm{C}$) were performed by a PCR machine running for 20 cycles. We then purified the DNA parts using magnetic beads and a 96 well plate
with integrated magenets \fullref{fig:figure2} (step2). A subsequent purification step using magneting attraction is performed to remove loose linkers and parts smaller than 100bp before the reaction wells are taken off the magnet to allow for
linker and DNA part annealing (Step 2). To check for successfull linear DNA parts, we run an gel-electrophoresis, prior to completing the assembly in a final assembly mixture of the four DNA parts and NEB CuSmart Buffer (10x). The assembly is 
also performed by the PCR machine at $50~^\circ\mathrm{C}$ for 45 minutes. The final plasmid is then transformed into E.coli, which inturn is then plated on two LB agar plates, one KanR-CamR and KanR only.
ligated and digested.  
\noindent   
\twocolend
\noindent
   {\footnotesize\resultfig{1.19}{\textbf{Results}. (a) The gel electrophoresis shows successfull signals at the expected range 0.5kb - 3kb. The image contains an overlay of the in silico digestion in benchlinq and the original.
   We can see that the predicted sized match the theoretical perfectly. (b) Total colony count, showing number of incorrect colonies (red), which is the amount of colonies showing the wrong flourecence profile (green or orange) and correct colleonies (grey). 
   Wells G1-G12 are individual group experiments, D4 is the total $\pm$ SEM (c) Accuracy of the assembly, by incorrect collonies divided by correct collonies Wells G1-G12 are individual groups, D4 is the total group accuracy and DC is the control.
   65.58\% $\pm$ 0.88\% (D4) and 1.8\% $\pm$ 6.2\% (DC)
   Figure created with GIMP and Biorender.com with adoption from from \cite{Storch2015}.}}
\twocolstart
\fullref{fig:figure2} (a) is showing a successfull digestion for all plasmids. The virtual digestions  where layed over the original gel image, perfectly aligning in silico and in vivo outcomes. Part b shows a high number of incorrect flourecence profiles accross all groups exept of
of G1, which is very close to the control. This results indicates a low accuracy (c) as seen in the diagram below. The total incauracy is 65.58 \% $\pm$ 0.88\% (D4) and 1.8\% $\pm$ 6.2\% (DC).

} 
\endinput