\def\introduction {
\section{Irreversable Circular Reaction}

 \lettrine[lines=2]{\color{color2}D}NA Assebly is key in genetic engineering, genome editing or auwith significan <sameet DNA sequences. Subsequent variatomation.
Restriction enyzme based methodsGolden Gate (multipart, Type II restriction enyzmes) \cite{Storch2015}, were still limited.
{\footnotesize\imgtaskA{\footnotesize \textbf{Irreversable Circular Reaction Pathway.} 
% stoichiometry 
Modular linkers consists of a 37bp linker section and 12bp long adapter section, designed as a DNA part internal prefix (iP) and suffix (iS).  
rts and at 3 prime orthogonally with eachother to form the reulting 45 bp long final linker. 
Figure created with Biorender.com with adoption from from \cite{Storch2015} }}
\noindent The linkers are reusable and rendering the assemly method as an automated high throughput mehtod \cite{Storch2015,Storch2017}.
\newline In this practical we are using two neutral linkers (L1, L2) and two iternal linker containing methylated prefix and suffix (LMiP,LMiS)\fullref{fig:figure2}.
The linkers are reusable and rendering the assemly method as an automated high throughput mehtod \cite{Storch2015,Storch2017}.
\noindent The linkers are reusable and rendering the assemly method as an automated high throughput mehtod \cite{Storch2015,Storch2017}.
\newline In this practical we are using two neutral linkers (L1, L2) and two iternal linker containing methylated prefix and suffix (LMiP,LMiS)\fullref{fig:figure2}.
The linkers are reusable and rendering the assemly method as an automated high throughput mehtod \cite{Storch2015,Storch2017}.
\noindent The linkers are reusable and rendering the assemly method as an automated high throughput mehtod \cite{Storch2015,Storch2017}.
\newline In this practical we are using two neutral linkers (L1, L2) and two iternal linker containing methylated prefix and suffix (LMiP,LMiS)\fullref{fig:figure2}.
The linkers are reusable and rendering the assemly method as an automated high throughput mehtod \cite{Storch2015,Storch2017}.
\noindent The linkers are reusable and rendering the assemly method as an automated high throughput mehtod \cite{Storch2015,Storch2017}.
\newline In this practical we are using two neutral linkers (L1, L2) and two iternal linker containing methylated prefix and suffix (LMiP,LMiS)\fullref{fig:figure2}.\\

\section{Dual Feedforward Circuit Analysis.}
The linkers are reusable and rendering the assemly method as an automated high throughput mehtod \cite{Storch2015,Storch2017}.
\twocolend
{\noindent\footnotesize\circuit{0.4}{\footnotesize \textbf{Reaction Pathway. Dual Feedforward Circuit.} Figure outlines digestion, purification, and annealing steps. Digestion and purification reactions are performed in seperate wells. Assembly is performed in one mixture.
Figure created with Biorender.com with adoption from \cite{Storch2015} }}
\twocolstart
{\noindent\footnotesize\imgtaskB_tikz{\footnotesize \textbf{Dual Feedforward Circuit.} Figure outlines digestion, purification, and annealing steps. Digestion and purification reactions are performed in seperate wells. Assembly is performed in one mixture.
Figure created with Biorender.com with adoption from \cite{Storch2015} }}
The Biopart Assembly Standard for Idempotent CLoning (BASIC) eliminates scars and specificity issues via othogonal linkers, which do not interfere with the DNA sequence, leading to scar-less cloning\cite{Storch2017}. 
Neutral linkers as a DNA part prefix (L1P, L2P) and suffix (L1S, L2S). The methylated linkers are designed as 2x2 set of Linkers for the Methylation prefix and suffix  (2xLMP,2xLMS), binding to the DNA parts. 
Each pair of LMP,LMS is further comprised of a orthogonal prefixes (-P) and suffixes (-S) which bind to eachother LMP-S, LMP-P and LMS-S, LMS-P. The adaptors are methylated at the 5 prime end for the prefix (LMP-P-A) and suffix (LMS-S-A) linkers. These methylations serve as a protection from DNA Ligase in the digestoin-ligations steps.
%3. Objective of Your Study
Our Objective is to evalutate the accuracy of a four part BASIC assembly process, consisting of Canamicin resistant DNA part (Kan-Ori); a Chloramphenicol resistance casette (CaR) and two flourecence cassttes (RFP, and GFP) \fullref{fig:figure2}.
First the DNA part containing plasmids are digested in four parallel reactions where DNA ligase cuts the plasmids at the BSaI site within the LMiP/S linkers \fullref{fig:figure2}.


\section{Machine Learning Implementation}
%
\twocolend 
{\footnotesize\resultfig{\textbf{Results}. (a) The gel electrophoresis shows successfull signals at the expected range 0.5kb - 3kb. The image contains an overlay of the in silico digestion in benchlinq and the original.
   We can see that the predicted sized match the theoretical perfectly. (b) Total colony count, showing number of incorrect colonies (red), which is the amount of colonies showing the wrong flourecence profile (green or orange) and correct colleonies (grey). 
   Wells G1-G12 are individual group experiments, D4 is the total $\pm$ SEM (c) Accuracy of the assembly, by incorrect collonies divided by correct collonies Wells G1-G12 are individual groups, D4 is the total group accuracy and DC is the control.
   65.58\% $\pm$ 0.88\% (D4) and 1.8\% $\pm$ 6.2\% (DC)
   Figure created with GIMP and Biorender.com with adoption from from \cite{Storch2015}.}}\\[0.3em]
\twocolstart

}
\endinput

